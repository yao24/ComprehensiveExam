\documentclass[11pt,a4paper]{article}
\usepackage[utf8]{inputenc}
\usepackage[T1]{fontenc}
\usepackage{amsmath}
\usepackage{amsfonts}
\usepackage[scale = 0.8]{geometry}
\usepackage{amssymb}
\usepackage{graphicx}
\usepackage{pdfpages}
\usepackage{hyperref}
\usepackage[round]{natbib}
\bibliographystyle{unsrtnat}
\usepackage{float}
\usepackage{subfig}
\usepackage{comment}
\setlength{\parindent}{0pt}
\title{Element-Based Galerkin method}
\author{Yao Gahounzo}

% insert month and year of your thesis, only:
%\date{October 4, 2021}
\date{\today} 

\begin{document}

    \maketitle
    
    Galerkin methods are numerical methods used to solve PDEs. This short note on Galerkin methods presents the Continous Galerkin (CG), where a free flux is assumed across the boundary. We described the CG method using the following diffusion equation
	
	\begin{equation}
	\label{eq:diff}
	    \dfrac{\partial q(x,t)}{\partial t} = \nu\nabla^2 q(x,t),
	\end{equation}
	
	\noindent with either Neumann or Robin boundary conditions. Let's divide the problem domain $\Omega\in\mathbb{R}$ into $N_e$ elements
	
	$$\Omega = \bigcup_{i=1}^{Ne}\Omega_e,$$
	
	We create an approximation $q_N^{(e)}(x,t)$ to $q(x,t)$ within each element $\Omega_e$ using 
	
	\begin{equation}
	\label{eq:17}
	    q_N^{(e)}(x,t) = \sum_{j=1}^{M_N} \psi_jq_j^{(e)}(t),
	\end{equation}
	
	where $M_N$ is the number of nodes in the element, $\psi_j$ is a basis function (e.g. Lagrange polynomials), $q_j^{(e)}$ is the expansion coefficient and the superscript $^{(e)}$ denotes the element index.
	
	\begin{figure}[H]
	    \centering 
	    \includegraphics[width=12cm]{CGpoly}
	    \caption*{Continuous Galerkin elements for polynomial order $N = 1$ and $N = 2$.}
    \end{figure} 
	
	We expand both sides of equation $(\ref{eq:diff})$ using the approximation $(\ref{eq:17})$, multiplying with the test function $\psi_i$ and integrate within each element 
	
	\begin{equation}
	    \int_{\Omega_e}\psi_i\dfrac{\partial q_N^{(e)}}{\partial t}d{\Omega_e} = \nu\int_{\Omega_e}\psi_i\nabla^2 q_N^{(e)}d{\Omega_e}
	\end{equation}
	
	Let us use now the product rule
	
	$$\nabla\cdot(\psi_i\nabla q_N) = \nabla\psi_i\cdot\nabla q_N + \psi_i\nabla^2 q_N$$
	
	After integrating and using divergence theorem we obtain
	
	\begin{equation}
	    \int_{\Omega_e}\psi_id{\Omega_e}\dfrac{d q_N^{(e)}}{dt} = \nu\int_{\Gamma_e}\psi_i\mathbf{n} \cdot\nabla q_N^{(e)}d{\Gamma_e} - \nu\int_{\Omega_e}\nabla\psi_i\cdot\nabla q_N^{(e)} d{\Omega_e}.
	    \label{eq:20}
	\end{equation}
	 $\Gamma_e$ represent the boundary of the element $\Omega_e$. Using the expansion $(\ref{eq:17})$, we get
	 
	 \begin{equation}
	    \sum_{j=1}^{M_N}\int_{\Omega_e}\psi_i\psi_jd{\Omega_e}\dfrac{d q_j^{(e)}}{dt} = \nu\int_{\Gamma_e}\psi_i\mathbf{n} \cdot\nabla q_N^{(e)}d{\Gamma_e} - \nu\sum_{j=1}^{M_N}\int_{\Omega_e}\nabla\psi_i\cdot\nabla \psi_j d{\Omega_e}q_j^{(e)}.
	    \label{eq:21}
	\end{equation}
	
	
	The first term on the right-hand side is used to apply the boundary conditions. Due to the continuity across element interfaces, the inter-element edges vanish when the direct stiffness summation is applied to the first term on the right-hand side.
	
	\section{Neumann boundary conditions}
	    
	    The Neumann boundary condition is generally described as follows
		\begin{equation}
			\label{eq:neu}
				\mathbf{n}\cdot\nabla q(x,t)  = h(x), \quad x\in \partial \Omega.
		\end{equation}
		
		\noindent Applying the Neumann condition $(\ref{eq:neu})$, equation $(\ref{eq:21})$ becomes 
	    
	    \begin{equation}
	    \sum_{j=1}^{M_N}\int_{\Omega_e}\psi_i\psi_jd{\Omega_e}\dfrac{d q_j^{(e)}}{dt} = \nu\int_{\Gamma_e}\psi_ih(x)d{\Gamma_e} - \nu\sum_{j=1}^{M_N}\int_{\Omega_e}\nabla\psi_i\cdot\nabla \psi_j d{\Omega_e}q_j^{(e)}.
	    %\label{eq:21}
	\end{equation}
	    
	    \noindent In the matrix form, we get 
	    \begin{equation}
		    M^{(e)}\dfrac{dq^{(e)}}{dt} = \nu B^{(e)} - \nu L^{(e)}q^{(e)}.
		\end{equation}
	
	where
	
	$$M^{(e)} = \sum_{j=1}^{M_N}\int_{\Omega_e}\psi_i\psi_jd{\Omega_e},$$
	
	$$L^{(e)} = \sum_{j=1}^{M_N}\int_{\Omega_e}\nabla\psi_i\cdot\nabla \psi_j d{\Omega_e},$$
	
	$$B^{(e)} = \int_{\Gamma_e}\psi_ih(x)d{\Gamma_e},$$
	
	$q^{(e)}$ is a vector that contains the $q_j^{e}$, and $i,j = 1,2,\ldots,M_N$.
	\section{Robin boundary condition}
	    
	    The general description of Robin boundary condition is given in below equation
	    
		\begin{equation*}
			\label{eq:4}
			\mathbf{n}\cdot\nabla q(x,t) + k q(x,t)  = g(x), \quad x\in \partial \Omega. 
		\end{equation*}
		
		\noindent Applying the above condition, equation $(\ref{eq:21})$ becomes
		
		\begin{align*}
		    \sum_{j=1}^{M_N}\int_{\Omega_e}\psi_i\psi_jd{\Omega_e}\dfrac{d q_j^{(e)}}{dt}& = \nu\int_{\Gamma_e}\psi_ig(x) - k\psi_iq(x,t)d\Gamma_e- \nu\sum_{j=1}^{M_N}\int_{\Omega_e}\nabla\psi_i\nabla\psi_jd{\Omega_e}q_j^{(e)}\\[0.3cm]
		    \sum_{j=1}^{M_N}\int_{\Omega_e}\psi_i\psi_jd{\Omega_e}\dfrac{d q_j^{(e)}}{dt}& = \nu\int_{\Gamma_e}\psi_i g(x)d\Gamma_e - k\nu\sum_{j=1}^{M_N}\int_{\Gamma_e}\psi_i\psi_jd\Gamma_eq_j^{(e)} - \nu\sum_{j=1}^{M_N}\int_{\Omega_e}\nabla\psi_i\nabla\psi_jd{\Omega_e}q_j^{(e)}\\
		\end{align*}
		
		In the matrix form, we get 
		\begin{equation}
		    M^{(e)}\dfrac{dq^{(e)}}{dt} = \nu B^{(e)} - \nu\left(kF^{(e)} + L^{(e)}\right)q^{(e)},
		\end{equation}
		
	where the new term $F^{(e)}$ is 
	
	$$F^{(e)} = \sum_{j=1}^{M_N}\int_{\Gamma_e}\psi_i\psi_jd\Gamma_e.$$
	 The direct stiffness summation is applied to move from the element reference equation to the global equation. For more detail the reader is referred to \cite{giraldo2020introduction}.
	 
	 \bibliography{refComp}
    
\end{document}